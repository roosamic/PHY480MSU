\documentclass{article}
\title{Project 2}
\author{Nat Hawkins, Victor Ramirez, Mike Roosa, Pranjal Tiwari}
\date{27 February, 2017}

\usepackage{relsize,makeidx,color,setspace,amsmath,amsfonts,amssymb}
\usepackage[table]{xcolor}
\usepackage{bm,ltablex,microtype}
\usepackage{placeins}
\usepackage{listings}
\usepackage[top = 1in, bottom = 1in, right = 1in, left = 1in]{geometry}
\usepackage[pdftex]{graphicx}
\usepackage{epstopdf}
\usepackage{inputenc}

\begin{document}
\maketitle

\begin{abstract}
	The goal of this project is to explore a model of quantum dots. We will be investigating the behavior of two electron in a 3-D simple harmonic potential while comparing the models with and without the particles interacting. To do this we will be solving the Schrödinger equation using the Jacobi method. 
	%some results to be added 
\end{abstract}

\section{Introduction}
Finding the eigenvalues of a matrix can give information on properties of a system, such as energies or spin. These properties can be useful to know when experiments are performed. Since we can predict the Hamiltonian of a system using the Shr$\ddot{o}$dinger equation. If we have a system with many particles, it is impractical to do so by hand, which is where computing eigenvalues from a matrix using computers becomes useful. Doing so will require a code that can find eigenvalues from a matrix, which is what the present work attempts to create.

\subsection{Mathematical Motivation}
The Hamiltonians that we will be concerned with will be in the form of a tridiagonal matrix and tridiagonal matricies are simple to get eigenvalues from, but if the matrix were 100x100, it would be too much to compute by hand.

\section{Solution}
\subsection{Setup}
We know that the hamiltonian is a tridiagonla matrix, where the diagonals are 2/$\hbar^2$ +V$_N$ and the elements on either side of the diagonals are -1/$\hbar^2$ for an NxN matrix:
\[
H=\begin{bmatrix}
\frac{2}{\hbar^2}+V_1 & -\frac{1}{\hbar^2}&0&0&...&0\\
 -\frac{1}{\hbar^2}&\frac{2}{\hbar^2}+V_2& -\frac{1}{\hbar^2}&0&...&0\\
0& -\frac{1}{\hbar^2}&\frac{2}{\hbar^2}+V_3& -\frac{1}{\hbar^2}&...&0\\
0&0&-\frac{1}{\hbar^2}&\frac{2}{\hbar^2}+V_4&...&0\\
...&...&...&...&...&-\frac{1}{\hbar^2}\\
0&0&0&0& -\frac{1}{\hbar^2}&\frac{2}{\hbar^2}+V_{N-1}
\end{bmatrix}
\]


\subsection{Jacobi Algorithm}

\subsection{Error}

\section{Conclusion}

\section{References}

%include article morten listed in the project description


\end{document}

